\documentclass[11pt]{article}

\usepackage[margin=1in]{geometry}
\usepackage{url}
\usepackage{ragged2e}
\usepackage{parskip} 
\usepackage{microtype}

\title{Introduction to Time Series: Fall 2024 \\ \smallskip
\large \Large Stat 153}
\author{}
\date{}

\begin{document}
\maketitle
\RaggedRight
\vspace{-50pt}

{\bf Instructor:}
Ryan Tibshirani, {\tt ryantibs@berkeley.edu} 

{\bf GSI:} 
Tiffany Ding, {\tt tiffany\_ding@berkeley.edu} 

{\bf Course website:}
\url{https://stat153.berkeley.edu/fall-2024/}

(See course website for lecture times, room, office hours, etc.)

\bigskip
In \emph{Introduction to Time Series}, we will cover the basics of time series 
analysis and prediction. This class will be mostly focused on computational and
practical aspects, with a limited emphasis on theory---for the most part, the 
theory we cover will be intended to help develop a better root understanding of
the nature of the topic at hand, say, the behavior or performance of a method
under idealized conditions. A stronger focus will be on developing a working 
``scientific intuition'' for time series problems and methods. Topics to be
covered will most likely include: characteristics of time series, regression, 
smoothing, forecasting, scoring, calibration, ensembling. 

\section*{Prerequisites}

Probability at the level of Stat 134 or Data 140 is required. Statistics at the
level of Stat 133 and 135 is recommended and may be taken concurrently. We will 
also assume a basic level of fluency with programming in R. Examples and
homework assignments will use the R programming language. (Counterpart examples
in Python may also be available, if we are able to pull it off.)

\section*{Topics}

In sligthly more detail, the topics that we will cover will likely include: 

\begin{itemize}
\item Measures of dependence
\item Stationarity
\item Regression
\item Smoothing
\item Spectral analysis
\item ARIMA models
\item ETS models
\item Advanced forecasters
\item Scoring and calibration
\item Ensembling
\end{itemize}

\section*{Evaluation}

Evaluation will be based on be five homeworks, one midterm exam, and one final
exam. The grading breakdown is as follows (each homework assignment is worth an
equal amount):    

\begin{itemize}
\item Homeworks: 50\% 
\item Midterm: 20\%
\item Final: 30\%
\end{itemize}

Details on the dates for the homework and exams will be provided on the course
website. More details on the exams will be forthcoming, later in the semester.

\section*{Homework}
 
The homeworks are structured to give you experience in written mathematical
exercises and programming exercises. As we may reuse problems from other,
similar courses that have been taught in the past, you \textbf{must not to copy,
  refer to, or look at} previous solutions in preparing your answers.

Also, while it is completely acceptable for you to collaborate with other
students in order to solve the homework problems, we assume that you will be
taking \textbf{full responsibility in terms of writing up your own solutions and
implementing your own code}. You must indicate on each homework the students
with whom you collaborated.

You will get a total of 5 late days that you can use for the homework
assignments that you can allocate in any way you choose across the
semester. (For example, you can use 5 days towards Homework 1; you can use 3
days towards Homework 1 and 2 days towards Homework 4; and so on.) Beyond that,
late homework will not be  accepted, except in the case of a true emergency
(sudden sickness, family problems, etc.). Just reach out to us (Instructor and
GSI) in the latter case, and we will figure something out.

\section*{Take care of yourself}

Take care of yourself. Do your best to maintain a healthy lifestyle this
semester by eating well, exercising, getting enough sleep, and taking some time
to relax. This will probably help you achieve your goals and cope with stress.

All of us benefit from support during times of struggle. You are not alone.
There are many helpful resources available on campus. You can find these linked
from the Academic Accomodations Hub 
\url{https://evcp.berkeley.edu/programs-resources/academic-accommodations-hub}.
\end{document}
